The one golden rule of space travel is simple: if you find a creepy
egg on a previously unexplored planet, DO NOT TAKE IT BACK WITH YOU.
Well, it seems Ensign R. Scott didn't get the memo, as after
a routine check on one of this system's planets, your crew finds
themselves running for their lives as a mysterious alien
rampages your ship!

Fortunately, there is protocol for such a situation. On your
\textbf{Ship Floorplan}, several stations are marked where you
can position a robotic guard to defend against the alien. 
Five guards must be deployed such that every point within the floorplan
is visible in a straight line from at least one guard.

An example for two guards is illustrated below. As long as one
guard is deployed within the top two rows, and the other guard is
placed within the bottom two rows, the entire area of the floor
is safely monitored. But if a guard is deployed on any of the
three stations in the middle row (spelling \texttt{BAD}),
there's no possible way for a second guard to monitor both
the top and bottom uncovered areas.

Your task is to locate all the guard stations on the ship that
should NOT be used when deploying only five guards. 
In addition to saving your crew from
certain death, the letters from these stations spell a hidden codeword!

\begin{center}
\begin{tikzpicture}[x=0.3in,y=0.3in]
\fill[white] (0,0) rectangle (7,6);
\draw[step=1,thin,black!20] (0,0) grid (7,6);
\hysExample
%\node[anchor=north] at (4,0) {2 guards, 3 unusable stations};
\node[draw,cross out] at (1,3) {};
\node[above left] at (1,3) {\footnotesize{B}};
\node[draw,cross out] at (2,3) {};
\node[above left] at (2,3) {\footnotesize{A}};
\node[draw,cross out] at (3,3) {};
\node[above left] at (3,3) {\footnotesize{D}};
\end{tikzpicture}

\begin{tikzpicture}[x=0.3in,y=0.3in]
\fill[white] (0,0) rectangle (7,6);
\fill[pattern=crosshatch, pattern color=black!40]
  (7,0.33) --
  (7,0) --
  (3,0) --
  (0,3) --
  (3,6) --
  (7,6) --
  (7,3.5) --
  (4,3.5) --
  (4,2.5) --
  cycle;
\fill[pattern=grid, pattern color=black!30]
  (2.2,5.2) --
  (4,2.5) --
  (7,2.5) --
  (7,0) --
  (3,0) --
  (0,3) --
  cycle;
\draw[line width=1pt,dotted] (2,4) -- (7,0.33);
\draw[line width=1pt,dotted] (5,1) -- (2.2,5.2);
\hysExample
\draw[fill=black!50] (1.9,3.9) rectangle (2.1,4.1);
\draw[fill=black!50] (4.9,0.9) rectangle (5.1,1.1);
\end{tikzpicture}
\hspace{0.2in}
\begin{tikzpicture}[x=0.3in,y=0.3in]
\fill[white] (0,0) rectangle (7,6);
\fill[pattern=crosshatch, pattern color=black!30]
  (0,3) --
  (3,6) --
  (7,6) --
  (7,4) --
  (4,3.5) --
  (4,2.5) --
  (7,2) --
  (7,0) --
  (3,0) --
  cycle;
\draw[line width=1pt,dotted] (1,3) -- (7,4);
\draw[line width=1pt,dotted] (1,3) -- (7,2);
\hysExample
\draw[fill=black!50] (0.9,2.9) rectangle (1.1,3.1);
\end{tikzpicture}
\end{center} 
