As early as 1500 B.C.E. the Fregian people played ball games as a national sport.
These games had deep religious significance and strict rules.
There were two kinds of balls, the white unmarked ball known as the hound, and the marked balls called the rabbits.
The balls were 2 meters in diameter, so perhaps it is more correct to refer to them as boulders.
The game was played in arenas with a fantastic variety of shapes and sizes.
The goal of the game was for the hound to drive each one of the rabbits into its hole at the edge of the arena: exactly one rabbit would be hit into each hole.
The hound would be launched at 45 degree angles, when it collides with a rabbit the rabbit continues in the direction it was hit.
Exactly one ball was allowed to go into each hole, the rabbits were not allowed to collide with each other, and no ball was allowed to hit a sharp corner.
One the other hand, players were encouraged to bounce the balls off of the walls, creating complex trajectories.
In her notes, the professor has sketched the layouts of many of these arenas.
Strangely, she also mentions having outrun a stray boulder.
She escaped unharmed, but her hat was never quite the same.
Perhaps the clue to deciphering the hidden message in the professor's notes can be gleamed from the winning sequence of rabbits and holes on these arenas.

Provided below is an example worked out by the teaching assistant.