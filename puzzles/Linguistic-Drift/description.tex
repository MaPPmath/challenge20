Language changes over time, sometimes in strange ways.
In the example below, Dr. Jonas was considering the changes in an ancient script over a period of 450 years.

\begin{center}
\fbox{
\begin{tikzpicture}[scale = .9, every node/.style={scale=.9}]
    \node[left] at (0,9) {\(+ \uparrow \uparrow + \Diamond \Searrow +\)};
    \node[circle, fill=black] at (.5,9) {};
    
    \node[right] at (10,9) {\(\times \equiv \bigcirc \Delta \times \times\)};
    \node[circle, fill=black] at (9.5,9) {};
    
    \node[left] at (0,8) {\(\rightarrow \downarrow \square \times \times \Diamond \uparrow \leftarrow\)};
    \node[circle, fill=black] at (.5,8) {};
    
    \node[right] at (10,8) {\(\times \Uparrow \times \times \Diamond \rightarrow \downarrow \times\)};
    \node[circle, fill=black] at (9.5,8) {};
    
    \node[left] at (0,7) {\(\times \bigcirc \Delta \rightarrow \times \times \square \times\)};
    \node[circle, fill=black] at (.5,7) {};
    
    \node[right] at (10,7) {\(+ \nabla \bigcirc \nabla \rightarrow \times \times \Diamond +\)};
    \node[circle, fill=black] at (9.5,7) {};
    
    \node[left] at (0,6) {\(\times \nabla \bigcirc \nabla + \square \leftarrow \bigcirc \times\)};
    \node[circle, fill=black] at (.5,6) {};
    
    \node[right] at (10,6) {\(+ \uparrow \Diamond \square \times \times \Diamond \square \bigcirc +\)};
    \node[circle, fill=black] at (9.5,6) {};
    
    \node[left] at (0,5) {\(+ = - \nabla \bigcirc \nabla \times +\)};
    \node[circle, fill=black] at (.5,5) {};
    
    \node[right] at (10,5) {\(\Searrow \Diamond + \Diamond \Nwarrow\)};
    \node[circle, fill=black] at (9.5,5) {};
    
    \node[left] at (0,4) {\(+ \bigcirc \Delta \times \times \Rightarrow \downarrow \square\)};
    \node[circle, fill=black] at (.5,4) {};
    
    \node[right] at (10,4) {\(+ \Leftrightarrow \Diamond \square = +\)};
    \node[circle, fill=black] at (9.5,4) {};
    
    \node[left] at (0,3) {\(\times \uparrow \square \square + \Diamond \Diamond \bigcirc \times\)};
    \node[circle, fill=black] at (.5,3) {};
    
    \node[right] at (10,3) {\(\Diamond \square \Nearrow \bigcirc + \nabla \bigcirc \nabla\)};
    \node[circle, fill=black] at (9.5,3) {};
    
    \node[left] at (0,2) {\(\square \square \rightarrow \uparrow \bigcirc + \bigcirc \Delta\)};
    \node[circle, fill=black] at (.5,2) {};
    
    \node[right] at (10,2) {\(\nabla + \square \bigcirc + + \bigcirc \Delta\)};
    \node[circle, fill=black] at (9.5,2) {};
    
    \node[left] at (0,1) {\(\nabla + \Diamond \bigcirc \times \times + \nabla \bigcirc \nabla\)};
    \node[circle, fill=black] at (.5,1) {};
    
    \node[right] at (10,1) {\(+ \nabla \bigcirc \nabla + \rightarrow \rightarrow \downarrow \square\)};
    \node[circle, fill=black] at (9.5,1) {};
    
    \node[left] at (0,0) {\(\times \leftarrow \rightarrow \Diamond \Diamond - - \times\)};
    \node[circle, fill=black] at (.5,0) {};
    
    \node[right] at (10,0) {\(+ \bigcirc \Delta \times \times \square \leftarrow +\)};
    \node[circle, fill=black] at (9.5,0) {};
    
    \draw [very thick] (.5,9) -- (9.5,8);
    \draw [very thick] (.5,8) -- (9.5,5);
    \draw [very thick] (.5,7) -- (9.5,7);
    \draw [very thick] (.5,6) -- (9.5,0);
    \draw [very thick] (.5,5) -- (9.5,9);
    \draw [very thick] (.5,4) -- (9.5,1);
    \draw [very thick] (.5,3) -- (9.5,6);
    \draw [very thick] (.5,2) -- (9.5,3);
    \draw [very thick] (.5,1) -- (9.5,2);
    \draw [very thick] (.5,0) -- (9.5,4);
\end{tikzpicture}
}
\end{center}

In the attached journal page, Dr. Jonas was looking at changes in that same script over a period of 600 years.
It looks like she hid a message in the connections between the words.
Here is a brief crash course in linguistics: you can expect a major change to occur every 150 years.
Keep an eye out for changes in boundaries, contractions, and expansions.
Good luck!
