The Skolem people of Mesopotamia had many myths and legends.
Of gret interest to Dr. Jonas was the story of queen Noether, famed for her ability to barter with traders and merchants.
You may have met a good haggler or two in your day, but they didn't have to deal with the strange ways of the Skolem Bazaar.
The Skolem people had no money, instead goods were exchanged for other goods.
Moreoever, at the start of each day the shopkeepers would declare their exchange rates.
They were notoriously stubborn and would not change these rates no matter what happened.

Dr. Jonas believed that Noether was real, and wanted to learn as much about her as possible.
Unfortunately, over time every story about Noether split into multiple versions.
In one, Noether entered the marketplace with one bag of spice.
\begin{itemize}
  \item The first shopkeeper declared that 1 bag of spice is equivalent to 4 shell bracelets and 1 clay cup. (\(S \leftrightarrow 4B + C\))
  \item The second shopkeeper declared that 1 shell bracelet is equivalent to 3 bags of spice and 3 clay cups. (\(B \leftrightarrow 3S + 3C\))
  \item The third shopkeeper declared that 1 clay cup is equivalent to 1 bag of spice and 1 shell bracelet. (\(C \leftrightarrow S + B\))    
\end{itemize}
There are two versions of the story:
\begin{enumerate}
  \item Noether entered the bazaar with 1 bag of spice and left with 7 clay cups.
  \item Noether entered the bazaar with 1 bag of spice and left with 7 shell bracelets.
\end{enumerate}

Dr. Jonas reasoned that the first version was possible and that the second was impossible.
Version one could happen the following way:
\begin{enumerate}
  \item Go to shopkeeper 1: \(S \rightarrow 4B + C\)
  \item Go to shopkeeper 2 and use 1 shell bracelet: \(4B + C = B + 3B + C = 3S + 3C + 3B + C = 3S + 3B + 4C\)
  \item Go to shopkeeper 3 to get a clay cups: \(3S + 3B + 4C = 3C + 4C = 7C\)
\end{enumerate}

So why is the second version impossible? Consider the total amount of bags of spice an clay cups: \(S + C\).
\begin{itemize}
  \item The first shopkeeper does not change this amount.
  \item The second shopkeeper increases/decreases this amount by 6.
  \item The third shopkeeper does not change this amount.
\end{itemize}
Since Noether starts with \(S + C = 1\), it can only become \(1, 7, 13,\) and so on.
So it is impossible for her to end up with exactly \(7\) bracelets.
In cluekeeper, this solution would be put in as PI.
If there were five variants which were possible, impossible, impossible, possible, and impossible, the solution would be PIIPI.

There are many more stories about Noether. If you can figure out which ones are possible and whiche are impossible, you might be able to deciper the mesage hidden in Dr. Jonas' journal.