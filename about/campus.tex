Here are some guidelines for local campuses on how to prepare for and
run the event.

\phSection{Schedule Template}

\begin{itemize}
  \item 0:00 - Staff arrives
  \item 0:15 - Team check-in
  \item 0:45 - Orientation
  \item 1:00 - Game Begins
  \item 4:00 - Game Ends
  \item 4:15 - Wrap-Up and Awards
  \item 4:30 - Dismissal
\end{itemize}

\phSection{Volunteers}

Only a handful of volunteers are required to run Game Control. We recommend
having 2-5 volunteers depending on the number of participating teams.

\phSection{Classroom Space}

A large \textbf{lecture hall} is recommended for running Check-in,
Orientation, and the Wrap-Up. Game Control can be stationed there during
the game as well, or another nearby classroom.

Each team should be given a separate \textbf{classroom} so that they may
openly collaborate with teammates without spoiling puzzles for other teams.
It is useful to affix \textbf{printed signs} on each classroom and Game
Control to help players navigate your space, as well as any additional
signage required to get around.

\phSection{Team Supplies}

\textbf{Scissors and tape} should be provided in each classroom.
In addition, \textbf{chalk or whiteboard markers} should be provided if
teams will have access to chalkboards or whiteboards in their room.
Some campuses provide \textbf{pencils and notepads} to players.
We recommend \textbf{inviting teams to bring additional supplies}, such as
graph paper, colored pencils, and simple calculators. 

Note that teams may also choose to bring laptops, cameras,
and so on, and must provide their own \textbf{smart phones/devices}
unless your campus provides them instead. 
We discourage campuses from banning the use of computers or the internet,
since a phone can be used to perform the same tasks, but also
do not suggest to explicitly recommend such items as they aren't required
to enjoy or be competitive in this game. The puzzles are
designed so that they generally cannot be solved using Google or 
brute force methods. (Savvy programmers might be able to write code to
help optimize their team's Bonus Puzzle solution, which is why the Bonus
Puzzle is essentially only used to break ties.)

\phSection{Copies}

All puzzles are designed to be printed/copied in \textbf{grayscale}, both
for the convience of campuses and for accessibility by players.
It is recommended to print copies for at least
\textbf{two more teams than you expect to participate} as extras, depending
on your access to last-minute copying.

You should have two separate PDFs: a complete \textbf{organizer book} that you're
currently reading, and a \textbf{player book} to be distributed to players.
We recommend printing \textbf{one organizer book per volunteer} in a binder,
and \textbf{one stapled player book per player}. 

\phSection{Check-in}

\textbf{Each player} should receive their \textbf{player book}, or at
least one book for every two players on a team.
They should also receive a \textbf{pencil} and \textbf{notepad} for use
during the game.

Teams' devices should be set up with the ClueKeeper app and Hunt Code
(but not the Start Code). Some campuses have players use their own phones; 
other campuses provide iPads.

Some campuses also choose to distribute other giveaways/swag/brochures
at registration. Many bookstores are willing to provide branded disposable
bags to help distribute materials.

\phSection{Orientation}

The \textbf{Rules} should be reviewed, and any questions from players
should be answered. In particular, boundaries for where players are allowed
to travel during the game should be established.

Once everyone is ready, present teams with the Start Code that will allow them
to begin solving puzzles and start the game timer. Teams should be instructed
to wait to enter it until they are settled in their HQ and ready to start solving.

\phSection{Gameplay}

As clues are unlocked in ClueKeeper, players will be able to solve puzzles
and input their solutions into ClueKeeper, earning points. Progress
may be monitored at ClueKeeper.com.

Campuses using Cluekeeper's GPS will have players traveling to locations
on campus to unlock Main Puzzles and Cryptic Puzzles. A campus map should
be provided mapping the numbers 000 to 999 to various locations on campus
(e.g. 000-049 is Building A, 050-099 is Building B, etc.). 
GPS enforcement can be turned off in case of rain, in which case the
three digit code may be entered from any location. An example of this map
for the University of South Alabama is included in this document; smaller
campuses may use less locations.

A volunteer should stand at the door of Game Control to ensure at most
one team is allowed in Game Control at all times, to avoid accidental
spoiling of puzzle data between teams. 

Recreational teams are allowed to ask for hints at Game Control at any time,
and may be assisted by their sponsor. We encourage new teams who may
not have a lot of math or puzzle experience to play recreationally.
Recreational teams are not eligible for trophies or awards, or should
be awarded separately from the competitive teams.

Competitive teams can ask for clarifications about the rules or how to 
use the ClueKeeper app, but generally
they should not receive hints outside of those provided by ClueKeeper.

Each team is allowed three submissions of the Bonus Puzzle. Generally this
puzzle should be judged by Game Control in front of the players to confirm
the validity of the submission. 
Only the best submission from each team is used. If the game has ended
with multiple teams in line for Game Control, all submissions for all teams
should be collected as quickly as possible and graded. Teams may not submit
multiple Bonus Puzzle solutions after the game has ended.

\phSection{Food}

Campuses that will be running the event through lunchtime are encouraged to
provide a \textbf{pizza lunch} for players. This lunch should not interrupt the
game; rather, players should be able to grab a bite to eat to have while they
continue to solve puzzles. In addition, \textbf{snacks}
(fruit, granola bars, etc.) and \textbf{drinks} (bottled water) are nice for
players to have access to during the game. Don't forget to provide
appropriate \textbf{plates, cutlery, napkins, and trashbags}.

This food can be distributed at a \textbf{central location near Game Control}
(but not inside Game Control's room where puzzles will be discussed).

\phSection{Wrap-Up and Awards}

At the end of the game, teams should straighten up their classrooms before
returning to Game Control for the Wrap-Up. \textbf{Trash bags} may be
provided for this purpose.

Teams should line up outside Game Control until results have been tabulated.
Once all results have been determined, teams may be seated inside Game Control.

Solutions need not be reviewed together; instead, the full game book can be provided
to teachers for follow-up/extension at a later date.

Awards for Recreational/Competitive teams are treated completely separately
if both Leagues are present.
\textbf{Certificates} should be distributed in 
random or alphabetical order to all teams placing
below 3rd place. A \textbf{3rd Place Certificate/Trophy} is then awarded.
After reminding the 1st place team to be respectful, a
\textbf{2nd Place Certificate/Trophy} is then awarded, followed by the
\textbf{1st Place Certificate/Trophy}. Opportunities for photographs should
be allowed during this process and after dismissal.

After awards are done, teams may be dismissed.

\phSection{Social Media}

Players/teachers/volunteers can be encouraged to tag \texttt{@MaPPmath}
and \texttt{\#Challenge20} on Twitter with non-spoiler posts/media during
and after the event. Everyone should be reminded that the game is played
at multiple locations across the country, and we do not want to spoil
future players on what's in store.

